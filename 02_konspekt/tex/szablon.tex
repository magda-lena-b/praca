\documentclass[a4paper,11pt,twoside]{report}
% KOMPILOWAĆ ZA POMOCĄ pdfLaTeXa, PRZEZ XeLaTeXa MOŻE NIE BYĆ POLSKICH ZNAKÓW

%-------------- DO: Kodowanie znakow, jezyk polski ----------------------------------------------------------
%------------------------------------------------------------------------------------------------------------
\usepackage[utf8]{inputenc}
\usepackage[MeX]{polski}
\usepackage[T1]{fontenc}
\usepackage[english,polish]{babel}

\usepackage{amsmath, amsfonts, amsthm, latexsym} % głównie symbole matematyczne, środowiska twierdzeń

\usepackage[final]{pdfpages} % inputowanie pdfa
%\usepackage[backend=bibtex, style=verbose-trad2]{biblatex}


%-------------- DO: Marginesy, akapity, interlinia ----------------------------------------------------------
%------------------------------------------------------------------------------------------------------------
\usepackage[inner=20mm, outer=20mm, bindingoffset=10mm, top=25mm, bottom=25mm]{geometry}



\linespread{1.5}
\allowdisplaybreaks

\usepackage{indentfirst} % opcjonalnie; pierwszy akapit z wcięciem
\setlength{\parindent}{5mm}


%-------------- DO: Zywa pagina -----------------------------------------------------------------------------
%------------------------------------------------------------------------------------------------------------
\usepackage{fancyhdr}
\pagestyle{fancy}
\fancyhf{}
% numery stron: lewa do lewego, prawa do prawego 
\fancyfoot[LE,RO]{\thepage} 
% prawa pagina: zawartość \rightmark do lewego, wewnętrznego (marginesu) 
\fancyhead[LO]{\sc \nouppercase{\rightmark}}
% lewa pagina: zawartość \leftmark do prawego, wewnętrznego (marginesu) 
\fancyhead[RE]{\sc \leftmark}

\renewcommand{\chaptermark}[1]{
\markboth{\thechapter.\ #1}{}}

% kreski oddzielające paginy (górną i dolną):
\renewcommand{\headrulewidth}{0 pt} % 0 - nie ma, 0.5 - jest linia


\fancypagestyle{plain}{% to definiuje wygląd pierwszej strony nowego rozdziału - obecnie tylko numeracja
  \fancyhf{}%
  \fancyfoot[LE,RO]{\thepage}%
  
  \renewcommand{\headrulewidth}{0pt}% Line at the header invisible
  \renewcommand{\footrulewidth}{0.0pt}
}

%-------------- DO: Naglowki rozdzialow ---------------------------------------------------------------------
%------------------------------------------------------------------------------------------------------------
\usepackage{titlesec}
\titleformat{\chapter}%[display]
  {\normalfont\Large \bfseries}
  {\thechapter.}{1ex}{\Large}

\titleformat{\section}
  {\normalfont\large\bfseries}
  {\thesection.}{1ex}{}
\titlespacing{\section}{0pt}{30pt}{20pt} 
%\titlespacing{\co}{akapit}{ile przed}{ile po} 
    
\titleformat{\subsection}
  {\normalfont \bfseries}
  {\thesubsection.}{1ex}{}


%-------------- DO: Spis tresci -----------------------------------------------------------------------------
%------------------------------------------------------------------------------------------------------------
\def\cleardoublepage{\clearpage\if@twoside
\ifodd\c@page\else\hbox{}\thispagestyle{empty}\newpage
\if@twocolumn\hbox{}\newpage\fi\fi\fi}

% kropki dla chapterów
\usepackage{etoolbox}
\makeatletter
\patchcmd{\l@chapter}
  {\hfil}
  {\leaders\hbox{\normalfont$\m@th\mkern \@dotsep mu\hbox{.}\mkern \@dotsep mu$}\hfill}
  {}{}
\makeatother

\usepackage{titletoc}
\makeatletter
\titlecontents{chapter}% <section-type>
  [0pt]% <left>
  {}% <above-code>
  {\bfseries \thecontentslabel.\quad}% <numbered-entry-format>
  {\bfseries}% <numberless-entry-format>
  {\bfseries\leaders\hbox{\normalfont$\m@th\mkern \@dotsep mu\hbox{.}\mkern \@dotsep mu$}\hfill\contentspage}% <filler-page-format>

\titlecontents{section}
  [1em]
  {}
  {\thecontentslabel.\quad}
  {}
  {\leaders\hbox{\normalfont$\m@th\mkern \@dotsep mu\hbox{.}\mkern \@dotsep mu$}\hfill\contentspage}

\titlecontents{subsection}
  [2em]
  {}
  {\thecontentslabel.\quad}
  {}
  {\leaders\hbox{\normalfont$\m@th\mkern \@dotsep mu\hbox{.}\mkern \@dotsep mu$}\hfill\contentspage}
\makeatother

%-------------- DO: Spisy tabel i obrazkow ------------------------------------------------------------------
%------------------------------------------------------------------------------------------------------------
\renewcommand*{\thetable}{\arabic{chapter}.\arabic{table}}
\renewcommand*{\thefigure}{\arabic{chapter}.\arabic{figure}}
%\let\c@table\c@figure % jeśli włączone, numeruje tabele i obrazki razem

%-------------- DO: Definicje, twierdzenia etc. -------------------------------------------------------------
%------------------------------------------------------------------------------------------------------------
\makeatletter
\newtheoremstyle{definition}%    % Name
{3ex}%                           % Space above
{3ex}%                           % Space below
{\upshape}%                      % Body font
{}%                              % Indent amount
{\bfseries}%                     % Theorem head font
{.}%                             % Punctuation after theorem head
{.5em}%                            % Space after theorem head, ' ', or \newline
{\thmname{#1}\thmnumber{ #2}\thmnote{ (#3)}}%  % Theorem head spec (can be left empty, meaning `normal')
\makeatother
%-------------- DO: Polski ----------------------------------------------------------------------------------
%------------------------------------------------------------------------------------------------------------
\theoremstyle{definition}
\newtheorem{theorem}{Twierdzenie}[chapter]
\newtheorem{lemma}[theorem]{Lemat}
\newtheorem{example}[theorem]{Przykład}
\newtheorem{proposition}[theorem]{Stwierdzenie}
\newtheorem{corollary}[theorem]{Wniosek}
\newtheorem{definition}[theorem]{Definicja}
\newtheorem{remark}[theorem]{Uwaga}

%-------------- DO: datkowe ---------------------------------------------------------------------------------
%------------------------------------------------------------------------------------------------------------

\usepackage{color}
%\usepackage{amssymb} %w pliku amssymb.sty zakomentowana 151i152 linijka: "%\DeclareMathSymbol{\lll}          {\mathrel}{AMSa}{"6E}"
\usepackage{wrapfig}
\usepackage{graphicx}
\usepackage{subfigure} %zeby mozna bylo kilka grafik obok siebie...

\setcounter{secnumdepth}{4}
%\titleformat{\subsubsection}

\usepackage{multirow}

%-------------- DO: Dowod -----------------------------------------------------------------------------------
%------------------------------------------------------------------------------------------------------------
%%\makeatletter
%\renewenvironment{proof}[1][\proofname]
%{\par
%  \vspace{-12pt}% remove the space after the theorem
%  \pushQED{\qed}%
%  \normalfont
%  \topsep0pt \partopsep0pt % no space before
%  \trivlist
%  \item[\hskip\labelsep
%        \sc
%    #1\@addpunct{:}]\ignorespaces
%}
%{%
%  \popQED\endtrivlist\@endpefalse
%  \addvspace{20pt} % some space after
%}
%
%\renewcommand{\qedhere}{\hfill \qedsymbol}
%\makeatother

%------------------------------------------------------------------------------------------------------------
%------------------------------------------------------------------------------------------------------------
%--------------------- POCZATEK -----------------------------------------------------------------------------
%------------------------------------------------------------------------------------------------------------
%------------------------------------------------------------------------------------------------------------

%--------------------- 00. USTAWIENIA UZYTKOWNIKA -----------------------------------------------------------
%------------------------------------------------------------------------------------------------------------

\newcommand{\tytul}{Zastosowania metod procesowania w języku polskim}
%\renewcommand{\title}{Missing value handling for classification tree creation and application}
%\newcommand{\type}{magisters} % magisters, licencjac
\newcommand{\supervisor}{dr~Grzegorz~Koloch}

\begin{document}
\sloppy

%\includepdf[pages=-]{titlepage}

%--------------------- 01. STRONA Z PODPISAMI AUTORA/AUTORÓW I PROMOTORA ------------------------------------
%------------------------------------------------------------------------------------------------------------
\thispagestyle{empty}\newpage
\null
\vfill
\begin{center}
\begin{tabular}[t]{ccc}
............................................. & \hspace*{100pt} & .............................................\\
podpis promotora & \hspace*{100pt} & podpis autora
\end{tabular}
\end{center}
%--------------------- 02. ABSTRAKTY -----------------------------------------------------------------------
%-----------------------------------------------------------------------------------------------------------


{
\begin{abstract}

\begin{center}
\tytul
\end{center}
Praca zawiera przegląd dostęPnych rozwiązań i metod dostępnych w zakresie NLP (natural language processing) dla języka polskiego wraz z ich zastosowaniem na przykładowym zbiorze danych tekstowych i analizą rezultatów. \\


\end{abstract}
}

\null\thispagestyle{empty}\newpage

%--------------------- 03. OSWIADCZENIE --------------------------------------------------------------------
%-----------------------------------------------------------------------------------------------------------

\null\thispagestyle{empty}\newpage

\null \hfill Warszawa, dnia ..................\\

\par\vspace{5cm}

\begin{center}
Oświadczenie
\end{center}

\indent Oświadczam, że pracę pod
tytułem ,,\tytul '', której promotorem jest \mbox{\supervisor}, wykonałam
samodzielnie, co poświadczam własnoręcznym podpisem.
\vspace{2cm}


\begin{flushright}
  \begin{minipage}{50mm}
    \begin{center}
      ..............................................

    \end{center}
  \end{minipage}
\end{flushright}

\thispagestyle{empty}
\newpage

\null\thispagestyle{empty}\newpage


%--------------------- 04. SPIS TRESCI ---------------------------------------------------------------------
%-----------------------------------------------------------------------------------------------------------
\pagenumbering{gobble}
\tableofcontents
\thispagestyle{empty}

\newpage % JEŻELI SPIS TREŚCI MA PARZYSTĄ LICZBĘ STRON, ZAKOMENTOWAĆ
% ALBO JAK KTOŚ WOLI WTEDY DWIE STRONY ODSTĘPU, DODAĆ \null\newpage

%--------------------- 05. ZASADNICZA CZESC PRACY ----------------------------------------------------------
%-----------------------------------------------------------------------------------------------------------
\null\thispagestyle{empty}\newpage
\pagestyle{fancy}
\pagenumbering{arabic}
% -> NOTE: JEŻELI Z POWODU DUŻEJ ILOŚCI STRON W SPISIE TREŚCI SIĘ NIE ZGADZA, TRZEBA ZMODYFIKOWAĆ RĘCZNIE
\setcounter{page}{11} 

%\chapter*{Wstęp}
%\markboth{}{Wstęp}
%\addcontentsline{toc}{chapter}{Wstęp}
\chapter{Wstęp}
%\markboth{}{Wstęp}
%\addcontentsline{toc}{chapter}{Wstęp}

W ostatnich latach obszar NLP rozwija się bardzo intensywnie. Powstają nowe rozwiązania pozwalające na zaawansowane przetwarzanie i analizę nieustrukturyzowanych danych jakimi są dane tekstowe. Coraz bardziej zaawansowane są modele pozwalające na interpretację jak również generowanie tekstu. \\

\section{podrozdzial}\label{miss_vs_class}

Niniejsza praca skupia się na~

\section{Cel i zakres pracy}
Celem niniejszej pracy jest  \\

\section{Przegląd rozdziałów}
W pierwszym rozdziale pracy  (ang. \textit{Missing Complately At Random, MCAR}), losowy (ang. \textit{Minatural language processing}) oraz \\

Podstawowe elementy przygotowywania danych do analiz związanych z przetwarzaniem języka naturalnego to:
\begin{itemize}
 \item tokenizacja, czyli podział tekstu na segmenty, najczęściej pojedyncze słowa,
 \item stemming ma na celu obcięcie wszystkich przyrostków i przedrostów aby zbliżyć słowo do podstawowej postaci,
 \item lematyzacja to przypisanie do każdego słowa jego formy podstawowej, która go reprezentuje, 
 \item tworzenie wektorów własnościowych (word embeddings) w uproszczeniu będących wektorową reprezentacją znaczenia danego słowa.
\end{itemize}


\chapter{Wstępne procesowanie danych}

\section{Źródło danych}
Zbiór na którym zostaną przeprowadzone analizy został pobrany z serwisu kaggle.com. Zawiera transkrypcje przemówień polskich polityków  z lat 1989 – 2019 oraz profil każdego z mówców. 

\section{Format danych}
Dane zostały udostępnione w formie bazy danych zawierającej dwie tabele. Pierwsza z nich zawiera szczegółowy profil każdego polityka. Do najistotniejszych informacji należą:

\begin{center}
\begin{tabular}{ |c|c| } 
 \hline
 Nazwa kolumny w zbiorze & Opis  \\ \hline
 full\_name & Imię i nazwisko \\  \hline
 cell7 & cell8 \\ 
 \hline
\end{tabular}
\end{center}

\section{Algorytm przypisania autora tekstu}
Występowanie brakujących wartości w \\ 

\section{Czyszczenie danych}
Wykonano następujące etapy czyszczenia danych: 
\begin{itemize}
 \item usunięto znaki specjalne 
 \item wykasowano tytuły przemówień pojawiające się na początku każdego fragmentu
 \item usunięty fragmenty tekstu w nawiasach (np. (Oklaski.), (Dzwonek.))
 \item Usunięto fragmenty identyfikujące mówcę, tj. fragment „Poseł Imię Nazwisko:”
\end{itemize}

Następnie przygotowano drugi zestaw tekstów, które oczyszczono jeszcze bardziej z fragmentów mało informacyjnych takich jak:
\begin{itemize}
 \item "Poseł Imię Nazwisko:"
 \item Panie Marszałku!
 \item Pani Marszałek!
 \item Wysoka Izbo!
 \item Panie Ministrze!
 \item Dziękuję bardzo.

\end{itemize}

Po tym etapie czyszczenia w bazie pozostało 272 217 fragmentów przemówień, z których 225 385 ma przypisanego autora. Oba zestawy tekstów z różnym poziomem oczyszczenia będą stosowane w różnych analizach.




%----------------------------------------------------------------------------------------------------------%----------------------------------------------------------------------------------------------------------%----------------------------------------------------------------------------------------------------------%----------------------------------------------------------------------------------------------------------
%----------------------------------------------------------------------------------------------------------%----------------------------------------------------------------------------------------------------------%----------------------------------------------------------------------------------------------------------%----------------------------------------------------------------------------------------------------------
%\section{Przykładowy podrozdział}

%\begin{definition}[Definicja]
%\textit{Definicją} nazywamy wypowiedź o określonej budowie, w której informuje się o znaczeniu pewnego wyrażenia przez wskazanie innego wyrażenia należącego do danego języka i posiadającego to samo znaczenie.
%\end{definition}

%\subsection{Przykładowy punkt}

%Poniżej punktu nie schodzimy.

%\begin{definition}[Równanie]
%\textit{Równaniem} nazywamy formę zdaniową postaci $t_1 = t_2$, gdzie $t_1, t_2$ są termami przynajmniej jeden z nich zawiera pewną zmienną.
%\end{definition}

%\begin{example}
%Przykładem równania jest:
%\begin{equation}
%2+2=4.
%\end{equation}

%Jeśli nie chcemy numerka przy równaniu, piszemy:
%\begin{equation*}
%2+2=4.
%\end{equation*}

%Można też:
%\[
%2+2=4.
%\]

%Równanie (\ref{rownanie}) jest fałszywe. Referencje (i kilka innych rzeczy) działają po dwukrotnym przekompilowaniu \TeX -a.

%\begin{equation}\label{rownanie}
%\int \limits_{0}^{1} x \; dx = \frac{3}{2}.
%\end{equation}

%\end{example}

%Twierdzenie \ref{Pitagoras} jest bardzo ciekawe.

%\begin{theorem}[Twierdzenie Pitagorasa]\label{Pitagoras}
%Niech będzie dany trójkąt prostokątny o przyprostokątnych długości $a$ i $b$ oraz przeciwprostokątnej długości $c$. Wówczas
%$$
%a^2 + b^2 = c^2.
%$$
%\end{theorem}

%\begin{proof}
%Dowód został zaprezentowany w \cite{Ktos} oraz \cite{Innyktos}. Czyli w sumie mogę napisać, że w %\cite{Ktos, Innyktos}. Albo że łatwo pokazać.
%\end{proof}

%\begin{corollary}
%Doszedłem do jakiegoś wniosku i daję temu wyraz.
%\end{corollary}




%\begin{remark}
%Lorem ipsum dolor sit amet, consetetur sadipscing elitr, sed diam nonumyeirmod tempor invidunt ut %labore et dolore magna aliquyam erat, sed diamvoluptua. At vero eos et accusam et justo duo dolores %et ea rebum.
%\end{remark}

%\begin{lemma}[Lemacik]
%Ten lemat jest nie na temat.
%\end{lemma}
%\begin{proof} Dowód przez indukcję.
%\end{proof}


%Lorem ipsum dolor sit amet, consetetur sadipscing elitr, sed diam nonumyeirmod tempor invidunt ut labore et dolore magna aliquyam erat, sed diamvoluptua. At vero eos et accusam et justo duo dolores et ea rebum. Stet clita kasd gubergren, no sea takimata sanctus est Lorem ipsum dolor sit amet.Lorem ipsum dolor sit amet, consetetur sadipscing elitr, sed diam nonumyeirmod tempor invidunt ut labore et dolore magna aliquyam erat, sed diamvoluptua. At vero eos et accusam et justo duo dolores et ea rebum. Stet clita kasd gubergren, no sea takimata sanctus est Lorem ipsum dolor sit amet.



%\section{Tabele i rysunki}

%\begin{table}% Koniecznie label po caption, inaczej jest zła numeracja
%\caption[Opis skrócony]{Opcje dodatkowe dla tabel i rysunków}
%\label{opcje}
%\centering
%\begin{tabular}{|c|p{0.8\textwidth}|}
%\hline
%symbol opcji & efekt \\ \hline
%\texttt{h} & bez przemieszczenia, dokładnie w miejscu użycia (uzyteczne w odniesieniu do niewielkich wstawek); raczej niestosowane \\
%\texttt{t} & na górze strony; stosowane najczęściej \\
%\texttt{b} & na dole strony \\
%\texttt{p} & na stronie zawierającej wyłącznie wstawki \\
%\texttt{!} & ignorując większość parametrów kontrolujacych umieszczanie wstawek, przekroczenie %wartosci, których może nie pozwolić na umieszczanie nastepnych wstawek na stronie \\ \hline
%\end{tabular}
%\end{table}

%W tablicy \ref{opcje} znajdują się opcje dodatkowe otoczeń \texttt{table} i \texttt{figure}.

%\begin{figure}[h!]

%\begin{center}
%    \setlength{\unitlength}{1mm}
%
%    \begin{picture}(40, 30)
%        \put(20,1){\line(0,1){20}} % linia
%
%        % dół
%        \put(20,1){\circle*{2}}
%        \put(25,1){0}
%
%        % góra
%        \put(20,21){\circle*{2}}
%        \put(25,21){1}
%    \end{picture}

%\end{center}
%\caption{Przykładowy rysunek, który można wygenerować w \LaTeX -u}
%\end{figure}


%Lorem ipsum dolor sit amet, consetetur sadipscing elit, sed diam nonumyeirmod tempor invidunt ut labore et dolore magna aliquyam erat, sed diamvoluptua. At vero eos et accusam et justo duo dolores et ea rebum. Stet clita kasd gubergren, no sea takimata sanctus est Lorem ipsum dolor sit amet.Lorem ipsum dolor sit amet, consetetur sadipscing elitr, sed diam nonumyeirmod tempor invidunt ut labore et dolore magna aliquyam erat, sed diamvoluptua. At vero eos et accusam et justo duo dolores et ea rebum. Stet clita kasd gubergren, no sea takimata sanctus est Lorem ipsum dolor sit amet.




%\chapter{Następny rozdział}

%Lorem ipsum dolor sit amet, consetetur sadipscing elit, sed diam nonumyeirmod tempor invidunt ut labore et dolore magna aliquyam erat, sed diamvoluptua. At vero eos et accusam et justo duo dolores et ea rebum. Stet clita kasd gubergren, no sea takimata sanctus est Lorem ipsum dolor sit amet.Lorem ipsum dolor sit amet, consetetur sadipscing elitr, sed diam nonumyeirmod tempor invidunt ut labore et dolore magna aliquyam erat, sed diamvoluptua. At vero eos et accusam et justo duo dolores et ea rebum. Stet clita kasd gubergren, no sea takimata sanctus est Lorem ipsum dolor sit amet.


%\section{Macierze}

%Prosta macierz:
%\[
%\begin{matrix}
%a & b & c & d \\
%d & e & f & g \\
%1 & 1 & 1 & 1
%\end{matrix}
%\]
%Macierz z nawiasami okrągłymi:
%\[
%A = 
%\begin{pmatrix}
%a & b & c & d \\
%d & e & f & g \\
%1 & 1 & 1 & 1
%\end{pmatrix}
%\]
%Macierz z nawiasami kwadratowymi:
%\[
%\begin{bmatrix}
%a & b & c & d \\
%d & e & f & g \\
%1 & 1 & 1 & 1
%\end{bmatrix}
%\]
%Można też ogólniejsze środowisko:
%\[
%\renewcommand{\arraystretch}{0.8}
%\begin{array}{ccc}
%1 & 0 & 0 \\
%0 & 1 & 0 \\
%0 & 0 & 1 \\
%\end{array}
%\]
%Nawiasy klamrowe:
%\[
%\left\{
%\renewcommand{\arraystretch}{0.8}
%\begin{array}{ccc}
%1 & 0 & 0 \\
%0 & 1 & 0 \\
%0 & 0 & 1 \\
%\end{array}\right\}
%\]

%\begin{definition}
%Niech $A\neq \emptyset$, $n \in \mathbb{N}$. Każde przekształcenie $f:A^n \rightarrow A$ nazywamy \textit{$n$-arną operacją} lub \textit{działaniem} określonym na $A$.
%0-arne operacje to wyróżnione stałe.
%\end{definition}


%\begin{definition}[Algebra]
%Parę uporządkowaną $(A,F)$, gdzie $A\neq \emptyset$ jest zbiorem, a $F$ jest rodziną operacji określonych na $A$, nazywamy \textit{algebrą} (lub \textit{$F$-algebrą}). Zbiór $A$ nazywa się \textit{zbiorem elementów}, \textit{nośnikiem} lub \textit{uniwersum} algebry $(A,F)$, a $F$ \textit{zbiorem operacji elementarnych}.
%\end{definition}

%\begin{proposition}
%Stwierdzam więc ostatnio, że doszedłszy do granicy, pozostaje mi tylko przy tej granicy biwakować albo zawrócić, możliwie też szukać przejścia czy wyjścia na nowe obszary.
%\end{proposition}


%--------------------- 06. BIBLIOGRAFIA ---------------------------------------------------------------------
%------------------------------------------------------------------------------------------------------------
% Bibliografia leksykograficznie wg nazwisk autorów
% Dla ambitnych - można skorzystać z BibTeX-a

\begin{thebibliography}{25}%jak ktoś ma więcej książek, to niech wpisze większą liczbę
% \bibitem[numerek]{referencja} Autor, \emph{Tytuł}, Wydawnictwo, rok, strony
% cytowanie: \cite{referencja1, referencja 2,...}
\bibitem{LR} R.J.A.Little, D.B.Rubin, \textit{Statistical Analysis With Missing Data}, Wiley Series in Probability and Statistics, 2002
\bibitem{JRQ} J.R. Quinlan, \textit{Induction of Decision Trees}, Machine Learning 1: Kluwer Academic Publishers, Boston 1986
\bibitem{BFOS} L.Breiman, J.H.Friedman, Richard A.Olshen, C.J.Stone, \textit{Classification and Regression Trees}, Chapman and Hall CRC, 1984
\bibitem{JWG} John W.Graham, \textit{Missing Data: Analysis and Design}, Springer, 2012
\bibitem{PC} P.Cichosz, \textit{Systemy uczące się}, Wydawnictwo Naukowo-Techniczne, Warszawa 2007
\bibitem{KC} J.Koronacki, J.Ćwik, \textit{Statystyczne systemy uczące się}, Wydawnictwo Naukowo-Techniczne, Warszawa 2005
\bibitem{FLMT} I.Fortes, L.Mora-Lopez, R.Morales, F.Triguero, \textit{Inductive learnings models with missing values}, Mathematical and Computer Modelling 44,2006, str.790-806
\bibitem{STP} Maytal Saar-Tsechansky, Foster Provost, \textit{Handling Missing Values when Applying Classsification Models}, Journal of Machine Learning Research 8, 2007, str.1625-1657
\bibitem{AA} Alan C. Acock, \textit{Working With Missing Values}, Journal of Marriage and Family 67, Listopad 2005, str.1012-1028
\bibitem{ZQLS} S.Zhang, Z.Qin, C.X.Ling, S.Sheng, \textit{Missing is Usefull: Missing Values in Cost-Sensitive Decision Trees}, IEEE Transactions on Knowledge and Data Enginiering, Vol.17, No.12, Grudzień 2005 
\bibitem{TJH} B.E.T.H.Twala, M.C.Jones, D.J.Hand, \textit{Good methods for coping with missing data in decision trees}, Pattern Recognition Letters 29, 2008, str.950-956
\bibitem{SO} J.L.Shafer, M.K.Olsen, \textit{Multiple imputation for multivariate missing-data problems: a data analyst's perspective}, Marzec, 1998 
\bibitem{XSCH} Y.Xiao, R.Song, M.Chen, H.I.Hall, \textit{Direct and Unbiased Muliple Imputation Methods for Missing Values of Categorical Variables},Journal of Data Science 10, 2012, str.465-481
\bibitem{PSP} F.Z.Poleto, J.M.Singer, C.D.Paulin, \textit{Missing data mechanisms and their implications on the analysis of categorical data}, Springer Science and Business Media, 2009
%\bibitem{LL} S.M.Lynch copyright by Scott M. Lynch March 2003 
\bibitem{PP} T.D.Pigott, \textit{A Review of Methods for Missing Data}, Educational Reserch and Evaluation Vol.7,No.4 , 2001, str.353-383
\bibitem{YY} Y.C.Yuan, \textit{Multiple Imputation for Missing Data: Concepts and New Development}, SAS Institute Inc., Rockville
\bibitem{WW} G.I.Webb \textit{The Problem of Missing Values in Decision Tree Grafting}, Department Of Mathematics and Computer Science University of Braunschweig, str.273-283
\bibitem{JLWZ} L.Jiang, C.Li, J.Wu, J.Zhu, \textit{A Combined Classification Algorithm Based on C4.5 and NB}, str.350-359
\bibitem{BGK} C.Borgelt, J.Gebhardt, R.Kruse, \textit{Concepts for Probabilistic and Possibilistic Induction of Decision Trees on Real World Data}
\bibitem{HSS} L.Hawarah, A.Simonet, M.Simonet, \textit{Dealing with Missing Values in a Probabilistic Decision Tree during Classification}, Institut d'Ingenierie et de l'Information de Sante Facul'e de Medecine 
\bibitem{KK} S.B.Kotsiantis, \textit{Decision trees: a recent overview}, Springer Science and Business Media B.V., 2011
\bibitem{NIH} M.L.Wallace, S.J.Anderson, S.Mazumdar, \textit{A stochastic multiple imputation algorithm for missing covariatedata in tree-structured survival analysis}, National Institutes of Health, Author Manuscript, December 2011, str.3004-3016
\bibitem{COHEN} J.Cohen, P.Cohen, S.G.West, L.S.Aiken, \textit{Applied Multiple Regression/Correlation Analysis for the Behavioral Sciences, Third Edition}, Lawrence Erlbaum Associates Publishers, 2003 
\bibitem{AF} A.Feelders, \textit{Handling missing data in trees: surrogate splits or statistical imputation?}, Tilburg University CentER for Economic Research, 1999
\bibitem{BM} G.E.Batista, M.C.Monard, \textit{An Analysis of Four Missing Data Treatment Methods for Supervised Learning}, University of Sao Paulo, 2003
\bibitem{DS} Y.Ding, J.S.Simonoff, \textit{An Investigation of Missing Data Methods for Classification Trees Applied to Binary Response Data}, Journal of Machine Learning Research (11), 2010, str.131-170
\bibitem{MS} M.Stone, \textit{Cross-validatory choice and assessment of statistical predictions}, Journal of the Royal Statistical Society B 36, 1974, str.111–147
\bibitem{SV} A.Svinivasan, \textit{Note on the location of optimal classifiers in ndimensional ROC space}, Technical Report PRG-TR-2-99, Oxford University Computing Laboratory, Oxford, 1999 
\bibitem{PD} F.Provost, P.Domingos, \textit{Well-trained PETs: Improving probability estimation trees}, Stern School of Business, New York University, 2001 
\bibitem{FS} L.A.Zadeh, \textit{Fuzzy sets}, Information and control 8, str.338-353, 1965 

\end{thebibliography}

\thispagestyle{empty}
\pagenumbering{gobble}


% --- 7. Wykaz symboli i skrótów - jeśli nie ma, zakomentować
%\chapter*{Wykaz symboli i skrótów}

%\begin{tabular}{cl}
%nzw. & nadzwyczajny \\
%* & operator gwiazdka \\
%$\widetilde{}$ & tylda
%\end{tabular}
%\\
%Jak nie występują, usunąć.
%\thispagestyle{empty}

% ----- 8. Spis rysunków - jeśli nie ma, zakomentować --------
\listoffigures
\thispagestyle{empty}
%Jak nie występują, usunąć.


% ------------ 9. Spis tabel - jak wyżej ------------------
\renewcommand{\listtablename}{Spis tabel}
\listoftables
\thispagestyle{empty}
%Jak nie występują, usunąć.


% 10. Spis załączników - jak nie ma załączników, to zakomentować lub usunąć

\chapter*{Spis załączników}
\begin{enumerate}
\item 00\_Kod\_sterujacy.R
\item 01\_Funkcje\_pomocnicze\_pakiety.R
\item 02\_Funkcje\_metoda\_surrogate\_splits.R
\item 03\_Funkcje\_metoda\_fractional\_instances.R
\item 04\_Wczytywanie\_zbiorow.R
\item 05\_Wybor\_parametrow\_drzewa.R
\item 06\_Symulowanie\_brakow\_danych.R
\item 07\_Badania\_wyniki.R
\item 08\_Rysowanie\_wykresow.R
\end{enumerate}
\thispagestyle{empty}


\end{document}
